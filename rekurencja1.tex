\documentclass[
fontsize=12pt %
,english %
,headinclude %
,headsepline % line between head an document text
%,BCOR=12mm %
]{scrbook} % twosided, A4 paper
\usepackage[T1]{fontenc}
\usepackage{polski}
\usepackage[utf8]{inputenc}
\usepackage[french,polutonikogreek,polish]{babel}
%\geometry{verbose,a4paper,tmargin=3cm,bmargin=2cm,lmargin=2cm,rmargin=2cm}
\usepackage{blindtext} % provides blindtext with sectioning
\usepackage{scrpage2} % header and footer for KOMA-Script
\usepackage{graphicx}
\clearscrheadfoot % deletes header/footer
\pagestyle{scrheadings} % use following definitions for header/footer
% definitions/configuration for the header
\rehead[]{\Large \textbf{BitAlgo Start}} % equal page, right position (inner)
\lohead[]{\Large \textbf{BitAlgo Start}} % odd page, left position (inner)
\cehead[]{Zadanie 2:\\Trzy podzbiory}
\cohead[]{Zadanie 2:\\Trzy podzbiory}
\lehead[]{\includegraphics[width=15mm]{logo.png}} % equal page, left (outer) position
\rohead[]{\includegraphics[width=15mm]{logo.png}}
% definitions/configuration for the footer
\cofoot[\pagemark]{\pagemark} % odd page, center position
\begin{document}
\vspace{50 mm}
\hspace{50 mm}
\newline
\par{\Large \textbf{Zadanie 2: Trzy podzbiory}} \\ \newline
Twoim zadaniem jest zaimplementowanie programu, który wypisze, ile jest podziałów zadanego zbioru 
liczb na trzy zbiory takich liczb, że w każdym podzbiorze łączna liczba jedynek użyta do zapisu elementów tego zbioru
w systemie dwójkowym jest jednakowa.
\\ \\
\par{\Large \textbf{Format wejścia}} \\ \newline
Pierwsza linia wejścia zawiera liczbe całkowitą $n$, $3 \leq n \leq 20$ - ilość elementów ciągu. W kolejnej linii znajduje się dokładnie $n$ liczb oddzielonych spacją, które stanowią zadany ciąg. Liczby te są z przedziału $[0; 10^9]$.
\\ \\
\par{\Large \textbf{Format wyjścia}} \\ \newline
Na wyjściu powinna zostać wypisana jedna liczba, która reprezentuje ilość takich podziałów.
\\ \\
\par{\Large \textbf{Przykład}} \\ \\
\begin{tabular}{ p{7cm} p{7cm} }
Dla danych wejściowych: \hspace{40mm}& Poprawną odpowiedzią jest \\
& \\
% define input here
6 \newline
2 3 5 7 11 15 \newline
&
% define output here
2 \newline
\\
\end{tabular}
\end{document}
